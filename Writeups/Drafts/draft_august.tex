\documentclass[]{article}
\usepackage{natbib}
\usepackage{hyperref}
\usepackage{amsmath}
\usepackage{setspace}
\usepackage{palatino}


%Importing bibtex
%opening


\title{The Geographic Bias of Growth}
\author{James Macek}
\onehalfspacing
\bibliographystyle{chicago}

%Colours for hyperreferencing
\hypersetup{
	colorlinks=true,
	linkcolor=blue,
	filecolor=blue,      
	urlcolor=blue,
	citecolor=blue,
	}


\begin{document}
\maketitle

\begin{abstract}


\end{abstract}

%%%%%%%%%%%%%%%%%%%%%%%%%%%___INTRODUCTION____%%%%%%%%%%%%%%%%%%%%%%%%%%%%%%%%%%%%%%%%
\section{Introduction}
\paragraph*{}
Test citation \citet*{urbstruct} \citet{tombezhu} \cite{hao2020} \citep{eckertpeters} \citet{delventhalglobenet}
\paragraph*{}
Contemporary models of spatial equilibrium imply that productivity growth, when uniform across locations, does not change the overall distribution of economic activity in space. In this paper, I argue that this
 feature masks an important mechanism by which growth concentrates economic activity even when the spatial distribution of productivity remains unchanged. 
\paragraph*{}
The theory rests on two ideas. Firstly, growth changes the consumption basket; shifting expenditures away from food and toward other goods (Deaton?). This manifests in general equilibrium as \textit{structural change}, rationalizing both differences in agricultural employment across countries and time. On the other hand, agriculture shapes geography insofar as the returns to concentrating it in space are dwarfed by other sectors. Taken together, these imply that workers leaving agriculture during the structural change process are absorbed by regions with high population density. 
\paragraph*{}
The link between structural transformation and urbanization has previously been explored in \citet{urbstruct}. In it, the high spatial concentration is driven by differences in land intensity and the spatial variance of productivity across sectors. As a consequence, population dense regions specialize outside of agriculture under trade and, conditional on specializing, have more dispersion in population densities. I build on this research in three ways. Firstly, I show theoretically that this effect need not rely on regional trade and specialization. Free trade mutes a consumption trade-off that underpins
 the choice to live in cities, which in the model have lower relative prices of non-agriculture products. Falling food expenditure shares make the penalty of locating in cities -- that is, high food prices -- less severe. 
\paragraph*{}  

 Both motives for specialization and consumption operate under the assumption that productivity differences across locations exist. As a second contribution, I extend the theory so that an uneven population distribution arises in a model where locations are identical ex ante. Growth in this environment also has the effect of reallocating workers toward densely populated regions.
 
 \paragraph*{}
 Thirdly, and most importantly, I quantify the magnitude of this effect by taking a spatial equilibrium model to Chinese data from 2000-2005. 


%%%BIBLIOGRAPHY%%%%%
\newpage
\bibliography{references.bib}

\end{document}
