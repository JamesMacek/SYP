\subsection{Exogenous productivity differences in autarky}
\paragraph*{}
The free trade model is purposefully stylized. By definition, prices are the same and so all backward linkages operate through the nominal wage that can be paid to workers in the more productive area. However, an empirically relevant feature of trade frictions is that they, along with migration costs, will induce spatial variation in the consumption share of agriculture $\omega_{i}$. This means that spatially uniform productivity growth will affect these locations differently.
\paragraph*{}
How can this idea potentially drive the spatial bias of growth? If spending on agriculture is falling, then consumers assign a smaller weight to agriculture in their consumption basket. This means that locations with higher food prices but lower prices in other sectors become more attractive, all else equal. Population-dense locations fall precisely into this category.
\paragraph*{}
To see this mechanism in action, I totally differentiate the indirect utility function with respect to prices. Let $-\frac{\partial P_{i, s}}{P_{i, s}} = g'_{s}$.  I normalize output per worker to one because the regions do not trade. Then, it can be shown that

\begin{equation}\label{differential}
	\frac{\partial U_{i}}{U_{i}} = \frac{\omega_{i}g'_{a} + (1-\omega_{i})g'_{n} }{1 - \omega_{i} + \omega_{i}\epsilon_{a}}
\end{equation}
Equation \eqref{differential} looks very similar to its homothetic counterpart but for the term in the denominator $1-\omega_{i} + \omega_{i}\epsilon_{a}$, which may vary across locations. Indeed, when $\epsilon_{a} = 1$, the formula collapses to the standard homothetic CES welfare formula. This additional factor is small in locations where $\omega_{i}$ is small whenever $\eta > 1$ or large otherwise. 


